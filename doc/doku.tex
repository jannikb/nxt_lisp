\documentclass[8pt]{article}

\usepackage[utf8]{inputenc}
\usepackage[ngerman]{babel}
\usepackage{amsmath}
\usepackage{nicefrac}
\usepackage{color}
\usepackage[hidelinks]{hyperref}

\usepackage[a4paper,top=2cm,left=4cm,right=4cm,bottom=2cm,marginparwidth=5cm]{geometry}
\usepackage{marginnote}
\usepackage[automark]{scrpage2}
\pagestyle{scrheadings}

\newcommand{\fatnote}[1]{\marginnote{\textbf{#1}}}
\newcommand{\leftnote}[1]{\reversemarginpar\fatnote{#1}}
\newcommand{\rightnote}[1]{\normalmarginpar\fatnote{#1}}

\newcommand{\defin}[1]{\noindent #1\vspace{0.3cm}}
\newcommand{\ldefin}[2]{\leftnote{#1}\defin{#2}}
\newcommand{\rdefin}[2]{\rightnote{#1}\defin{#2}}

\newcommand{\todo}[1]{\textcolor{red}{\textbf{TODO}: #1}}

\newcommand{\coursename}{\@empty}
\newcommand{\groupno}{\@empty}

\newcommand{\course}[2]{\renewcommand{\coursename}{#1}\renewcommand{\groupno}{#2}}
\newcommand{\beginsheet}{\clearscrheadfoot\ihead[]{Kurs: \coursename}\ohead[]{Gruppe \groupno}\ofoot[]{\pagemark}\ifoot[]{Dieses Dokument ist Teil der Dokumentation}}

\newcommand{\email}[1]{\href{mailto:#1}{#1}}


%%%%%%%%%%%%%%%%%%%%%%%%%%%%%%%%%%%%%%%%%%%%%%%%%%%%%%%%%%%%%%%%%%%%%%%%%%%%%%%%%%%%%%
%% Oberhalb dieses Blocks nichts ändern
%%%%%%%%%%%%%%%%%%%%%%%%%%%%%%%%%%%%%%%%%%%%%%%%%%%%%%%%%%%%%%%%%%%%%%%%%%%%%%%%%%%%%%


%% TODO: Hier die fehlende Gruppennummer einfügen
\course{Lisp Kurs -- Roboterprogrammierung in Lisp}{4}


\begin{document}

\beginsheet

\title{SUTURObot}
\author{Jannik Buckelo, Simon Stelter}
\date{13.07.2014}
\maketitle

\newpage

\tableofcontents

\newpage

\begin{abstract}
Dokumentation des SUTURObots der im Rahmen des Lisp Tutorials an der Universit"at Bremen erstellt wurde. Zur Konstruktion des Roboters wurde ein LEGO Mindstorm verwendet, der mittels eines Smartphones gesteurt werden kann. Die Steuerung der Motoren und das Auswerten der Sensoren wird dabei von einem Common Lisp-Programm geregelt.	
\end{abstract}

\section{einleitung} simon
alte idee ging nicht
neue idee

\section{SUTURObot v1}

\subsection{lösungsansatz} simon

\subsection{architektur}

\subsubsection{allgemein} jannik

\subsubsection{handy} simon
app 
umrechnen

\subsection{Probleme} simon

\section{SUTURObot v2}

\subsection{lösungsansatz + idee} jannik
Nachdem unser SUTURObot v1 aus vorher beschriebenen Gr"unden nicht funktionert hat, mussten wir uns ein neues Konzept ausdenken. Wir wollten dabei m"oglichst viel von unserem bis dahin erstellten Code und Infrastruktur wiederverwenden. Also haben wir uns "uberlegt die bereits bestehende Android-App und Funktionen zur Lagebestimmung des Handys zu verwenden, um den Roboter zu steuern.

Wir haben dazu eine holonome Basis f"ur den Roboter gebaut und die App mit zwei zus"atzlichen Buttons ausgestattet, sodass der Effort der R"ader mittels kippen des Smartphones geregelt werden kann und die Drehung der R"ader mittels Buttonklick. Au"serdem haben wir vier Drucksensoren an den Seiten des Roboters zur Erkennung von Kollisionen angebracht. Sollte es zu einer Kollision kommen soll der Roboter selbst"andig die Bewegung in Richtung des Hindernisses einstellen und ein St"uck von dem Hindernis wegfahren.

\subsection{architektur}
top down

\subsubsection{allgemein} jannik



\subsection{sensorik} jannik

\subsubsection{visual} simon

\subsection{Probleme} simon

\end{document}








